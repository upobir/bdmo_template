\documentclass[article, 12pt, a4paper, onesize]{memoir}
\usepackage{bdmo}

\begin{document}


%%%%%%%%%%%%%%%%%%%%%%%%%%%%%%%%%%%%%%%%%%%%%%%%%%%%%%
%% Usage:                                           %%
%%  \printtitle{<Contest Name>}{<Category>}         %% 
%%  Examples of <Contest Name>:                     %%
%%      ``বাছাই পর্ব: ঢাকা জেলা'',                                   %%
%%      ``বরিশাল আঞ্চলিক গণিত উৎসব''                          %%
%%  \venue{<Date>}{<School Venue>}                  %%
%%   Use the Problems environment                   %%
%%   To add problems, just add                      %%
%%   \problem{<বাংলায় প্রশ্ন>}{<ইংরেজিতে প্রশ্ন>}{<উত্তর>}   %%
%%%%%%%%%%%%%%%%%%%%%%%%%%%%%%%%%%%%%%%%%%%%%%%%%%%%%%

\printtitle{২০২১}{বাছাই পর্ব: ঢাকা জেলা}{প্রাইমারি}{১ ঘন্টা}{২০২০}
\venue{ডিসেম্বর ১৪, ২০২০}{ঢাকা রেসিডেনশিয়াল মডেল কলেজ}

\begin{Problems}
	\problem{এখানে বাংলায় সমস্যার বিবরণ দিতে হবে।}{The problem statement in English goes here.}{আর এখানে যাবে উত্তর।}
	\problem{এই সারির নিচেই একটা নমুনা প্রশ্ন আছে।}{A sample problem is right below this row.}{}
    \problem{\(f\) হলো ধনাত্মক পূর্ণসংখ্যার সেট থেকে ধনাত্মক পূর্ণসংখ্যার সেটে এমন একটা ফাংশন যেন যেকোনো পূর্ণসংখ্যা \(n\)-এর জন্য যদি \(x_1, x_2, \cdots , x_s\) সংখ্যাগুলো \(n\)-এর সবগুলো ধনাত্মক উৎপাদক হয়, তাহলে \(f(x_1)f(x_2)\cdots f(x_s)=n\)। \(f(343)+f(3012)\)-এর সম্ভাব্য সকল মানের যোগফল নির্ণয় করো।}
    {Let \(f\) be a function from the set of positive integers to the set of positive integers such that for each positive integer \(n\), if \(x_1, x_2, \cdots , x_s\) are all the positive divisors of \(n\), then \(f(x_1)f(x_2)\cdots f(x_s)=n\). Find the sum of all possible values of \(f(343)+f(3012)\).}
    {8}
    \problem{মোটামুটি সহজ প্রশ্ন}{An easy-ish problem}{}
    \problem{মোটামুটি সহজ প্রশ্ন}{An easy-ish problem}{}
    \problem{মিডিয়াম ডিফিকাল্টির প্রশ্ন}{A problem of medium difficulty}{}
    \problem{মিডিয়াম ডিফিকাল্টির প্রশ্ন}{A problem of medium difficulty}{}
    \problem{কঠিন প্রশ্ন}{A hard problem}{}
    \problem{কঠিন প্রশ্ন}{A hard problem}{}
    \problem{খুবই কঠিন প্রশ্ন}{A very hard problem}{}
\end{Problems}

\end{document}
